\documentclass[12pt]{article}

\usepackage{xcolor}
\usepackage{float}
\usepackage{subcaption}
\usepackage{graphicx}
\usepackage[tmargin=0.3in]{geometry}
\usepackage{url}
\usepackage[hidelinks]{hyperref}

\newcommand\ytl[2]{
\parbox[b]{8em}{\hfill{\color{black}\bfseries\sffamily
#1}~$\cdots\cdots$~}\makebox[0pt][c]{$\bullet$}\vrule\quad \parbox[c]{6cm}{\vspace{7pt}\color{black}\raggedright\sffamily
#2\\[7pt]}\\[-3pt]}

\usepackage{avant}
\begin{document} \titlepage{}

\begin{figure} \centering
	\begin{subfigure}{0.15\textwidth} \flushleft{}
		\includegraphics[width=\linewidth]{foss_logo}
	\end{subfigure}%
		\begin{subfigure}{0.4\textwidth} \centering
		\includegraphics[width=\linewidth]{maxima}
	\end{subfigure}%
		\begin{subfigure}{0.15\textwidth} \flushright{}
		\includegraphics[width=\linewidth]{iist_logo}
	\end{subfigure}
\end{figure}

\vspace*{-1cm}

\centering { \textbf{\LARGE FOSS Group, IIST}}\\ { \textbf{\Large
Introduction to Maxima}\\ \emph{Indian Institute of Space Science and Technology,
Thiruvananthapuram}

\centering

\vspace{0.5cm} {
  \begin{tabular}{rl}
    \textbf{DATE }:& Wednesday, March 22,2017\\
    \textbf{VENUE }:& C-104, D4-building, IIST\\
    \textbf{TIME }:& 3:00 p.m. to 4:00 p.m.\\
    \textbf{SPEAKER }:& Nidish Narayanaa B\\
    &(B.Tech Sem VIII)
  \end{tabular}}
\begin{center}
  Maxima is a Free and Open Source Computer Algebra System
  (CAS). Written in Common LISP, the software focuses mainly on
  the manipulation of symbolic expressions, algebra and calculus.\\
\emph{  The present discussion is an attempt to provide a tutorial
  introduction to using the platform for engineering and scientific
  purposes. }
\end{center}
The following is a list of the sections that will be dealt with in the
session. 
\begin{table}[H] \centering
  \begin{minipage}[t]{.7\linewidth}
    \color{gray}
    \rule{\linewidth}{1pt}
    \ytl{1}{Introduction}
    \ytl{2}{Basics}
    \ytl{3}{Plotting}
    \ytl{4}{Solving Equations}
    \ytl{5}{Matrices}
    \ytl{6}{Calculus}
    \ytl{7}{Symbolic Manipulation}
    \ytl{8}{Integration with \LaTeX{}}
    % \bigskip
    \rule{\linewidth}{1pt}%
  \end{minipage}%
\end{table}
\textbf{P.S.: Participants are expected to come with a computer having
  a working maxima version installed.} \\
\begin{flushleft}
  \textbf{Useful Links:}
  \begin{enumerate}
  \item \url{http://maxima.sourceforge.net/}
  \item \url{https://sourceforge.net/projects/wxmaxima/} (recommended
    front end)
  \item \url{http://web.csulb.edu/~woollett/} (great resources)
  \end{enumerate}
\end{flushleft}
\end{document}
