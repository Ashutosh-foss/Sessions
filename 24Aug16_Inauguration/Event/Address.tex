% Created 2016-08-24 Wed 14:10
\documentclass[11pt]{article}
\usepackage[latin1]{inputenc}
\usepackage[T1]{fontenc}
\usepackage{fixltx2e}
\usepackage{graphicx}
\usepackage{longtable}
\usepackage{float}
\usepackage{wrapfig}
\usepackage{rotating}
\usepackage[normalem]{ulem}
\usepackage{amsmath}
\usepackage{textcomp}
\usepackage{marvosym}
\usepackage{wasysym}
\usepackage{amssymb}
\usepackage{hyperref}
\tolerance=1000
\author{Nidish Narayanaa}
\date{\today}
\title{Address}
\hypersetup{
  pdfkeywords={},
  pdfsubject={},
  pdfcreator={Emacs 24.5.1 (Org mode 8.2.10)}}
\begin{document}

\maketitle

\section{Welcome}
\label{sec-1}
Director Dr. VK Dadhwal, deans, faculty members and dear friends, Good
afternoon all. Thank you for joining us today. I am Nidish, a final
year BTech student currently pursuing Aerospace Engineering.

We are here for the inaugural session of the FOSS group of IIST. FOSS
is short for Free and Open Source Software - in order to go into what
it actually stands for, we will have to look into what is known as the
GNU project, or the Software Freedom movement. We now have Gowtham,
speaking to us about what the Free Software movement is all about and
how relevant it is in the current scenario.

\section{FOSS}
\label{sec-2}
GNU, standing for GNU's Not Unix, represents a rebellion against
proprietary software copyrights placing restrictions on
redistribution, sharing and general usage of software from the
user's side. The project was initiated by Richard Stallman in 1983,
the then to-be-founder of the Free Software Foundation. He defines
Free Software as software that respects the user's freedom and the
social solidarity of his community.

The word Free in FOSS pertains to freedom, not price. It is a common
refrain, "Free as in Freedom; not free beer". Realizing the
implications of such freedom to the modern scientific community. we
have come up with IIST's own FOSS group.

The advantage with FOSS is that anybody is authorized to distribute
and/or publish the modified or unmodified source code since it is
openly shared. Richard Stallman also came up with the GNU Public
License, which acted as a "copyleft" in providing the users with
necessary freedom rather than restricting it from them through
copyrights. 

The GPL offers fouur kinds of freedom to the users:
\begin{itemize}
\item Freedom to run the program for any purpose
\item Freedom to study how the program works
\item Freedom to redistribute copies
\item Freedom to improve the program and also release improvements to
the public
\end{itemize}

The above will be kind of the driving motto for the FOSS group of
  IIST. The main idea is that knowledge is a common heritage of 
  mankind and thus software, a form of knowledge, has to be free and
  accessible to all.

Realizing the importance, in the March of 2015, the Government of
India adopted a comprehensive and supportive open source
policy. The policy requests all governmental organizations to
prefer the use of open source software in all central government
departments and ministries, with use of closed source software
only being considered as an exception with sufficient
justification.

I would like to conclude by noting that IIST's FOSS group, will be
India's first Space research oriented FOSS chapter and the avenues
to be explored are numerous.

\section{Interlude}
\label{sec-3}
Thank you Gowtham. In order to drive home the idea of "free
software - free world" in IIST's perspective, we have Nikhil,
another final year BTech student from out batch.

\section{Nikhil's Part}
\label{sec-4}

\section{Interlude}
\label{sec-5}
Thank You Nikhil. I now call upon our Director, Dr. Vinay Kumar
Dadhwal to grace us with his address.

\section{Director's Address}
\label{sec-6}

\section{Github Repo Creation}
\label{sec-7}
Thank You Sir. In order to kick things off, we want to create a
repository on github, an open source online Git repository hosting
service. 

Git is an open source version control system used to track and
monitor changes in large software programming and other projects. We
will have a detailed session on version control systems very
soon. The repository is intended to be the central resource base of
the FOSS group of IIST - it will contain software tutorials
prepared/adapted by members and serve as a reference containing
information of current projects and undertakings of the group and
some other miscellany pertaining to the group.

We request our director, Dr. V K Dadhwal to initiate IIST's FOSS
chapter by creating the repository on github.

\section{Github Repo Creation}
\label{sec-8}

\section{Interlude}
\label{sec-9}
Thank You Sir. We are sure the repository will stand witness to many
projects the FOSS group will undertake.

In order to give the audience a more informed picture of FOSS based
research in Aerospace Engineering, we have Dr. Manoj T Nair,
Associate Professor from the department of Aerospace Engineering.

\section{Dr. Manoj Nair's address}
\label{sec-10}

\section{Interlude}
\label{sec-11}
Thank You Sir.

We now have Dr. Manoj B S, associate professor from the department
of Avionics, giving us an idea of how FOSS tools may supplement
research in computer science and other avionics disciplines.

\section{Dr. Manoj B S's address}
\label{sec-12}

\section{Interlude}
\label{sec-13}
Thank You Sir.

To have a feel of what is in store from the group and get a good
idea of the upcoming activities, we have Dr. Devendra Ghate, a
faculty from Aerospace engineering who has worked very closely with
us in setting up this group.

\section{Dr. Ghate's address}
\label{sec-14}

\section{Concluding Remarks by Sudhanshu}
\label{sec-15}
% Emacs 24.5.1 (Org mode 8.2.10)
\end{document}